Letting a circle of ratio $a$ roll on the $x$-axis, results in a curve,
called a Cycloid.

\begin{Definition}
  \textbf{Cycloids}.
  
By a Cycloid, with ratio $a$ and phase $\gamma$,
$C(a,\gamma)$, we understand 
the curve parametrisized by:
\[
 \Vector{r}(t)
  =
  at\Vector{i}+a\Vector{j}
  -a \Vector{p}(t+\gamma)
\]

The Cycloid $C(1,-\pi/3)$ is shown in Fig. \ref{fig:cycloid}.
\end{Definition}


\Figure
{Trochoids/Cycloid.png}
{Cycloid}
{
  The Cycloid $C(1,-\pi/3)$, (grey curve).
  Included is, calculated using numerical derivation:
  the Cycloid's evolute (orange curve),
  the curvature vecor (magenta), the osculating circle (green)
  as well as the coordinate systems: World (white), rolling circle (red)
  and the accompanying, $(\Vector{r},\Vector{t},\Vector{n})$ (yellow).
  In the figures to follow, we shall adopt this colouration scheme.
}
{fig:cycloid}

Fig. \ref{fig:cycloid} seems to indicate, that:  The Evolute of the Cycloid,
is another Cycloid. We shall prove this in the remainder of this section.


   Scaling the rolling vector, $b\Vector{p}(t)$, $b \in \mathbb{R}$,
   we obtain another curve, called
   a Trochoid. As for the Cycloid,  we allow for a phase, $\gamma$, 
   thus parametrizising:
   

\begin{Definition}
  \textbf{Trochoids}.
  
   By a $\gamma$-fased Trochoid, $T(a,b,\gamma)$, we understand the curve
   parameterized by:
   
\[
  \Vector{r}(t)
 =
  \begin{pmatrix}
   x(t)\\y(t)
  \end{pmatrix}
  =
  \begin{pmatrix}
   at-b\sin{(t+\gamma)}\\a-b\cos{(t+\gamma)}
  \end{pmatrix}
  =
  \]
  \[
  a
  \begin{pmatrix}
   t\\1
  \end{pmatrix}
   -
   b
  \begin{pmatrix}
  \sin{(t+\gamma)}\\\cos{(t+\gamma)}
  \end{pmatrix}
 =
 \]
 \[
  at\Vector{i}+a\Vector{j}
  -b \Vector{p}(t+\gamma)
\]
\end{Definition}

\Figure
{Trochoids/Trochoid_0-5.png}
{Trochoid}
{
  The Cycloid $T(1,0.5,-\pi/3)$
}
{fig:trochoid}





The Trochoid $T(a,-a,\gamma)$, is a rephased Cycloid: $C(a,\gamma+\pi)$.
Omitting the phase parameter, we assume it $0$: $T(a,b)=T(a,b,0)$
and $C(a)=C(a,\gamma)$.
Before continuing, we note that the vector $\Vector{p}(t)$, behaves similarly
to the vectors $\Vector{e}$ and $\Vector{f}$ in the previous section. 
It's orthogonal complement, $\Vector{q}$, is:
\[
  \Vector{q}(t)
  =
  \begin{pmatrix}
   -\cos{t}\\\sin{t}
  \end{pmatrix}
\]

Omitting the tedious parameter, $t$, clearly: 
$\Vector{\widehat{q}}=-\Vector{p}$, and for the derivatives:
\[
 \Vector{p}'
 =
 -\Vector{q},
 \qquad \qquad
 \Vector{q}'
 =
 \Vector{p}
\]

Considering the family of Trochoids, we shall calculate the curvature, $\kappa(t)$,
the curvature radius, $\rho(t)$, as well as the curvature vector,
$\rho \Vector{n}$
and the parametrization of the Trochoid evolute, 
$\Vector{c}=\Vector{r}+\rho \Vector{n}$. In fact, we will show that
the evolute of a Cycloid, is a translated Cycloid, whereas this is \emph{not}
the case for the general Trochoid, $b \neq 0$.

Deriving once:

\[
 \Vector{r}'
 =
 a \Vector{i}
 +
 b \Vector{q}
\]

Note, that $\Vector{r}'(t)=0$, implies 
$\sin{t}=0$, that is: $t=p \pi, ~  p \in \mathbb{Z}$.
This satisfied, $\cos{p \pi}=(-1)^p=-a/b$. That is,
for $b=a$ (the Cycloid), the irregular points
are at $t=(2n+1)\pi$. In the case
$b=-a$, which is another Cycloid, the irregular points
are at $t=2n \pi$. In the case of a 'true' Trochoid, $b \neq \pm a$, all points
are regular.

The second derivative is:
\[
 \Vector{r}''
 =
 b \Vector{p}
\]

Taking the orthogonal complement of the derivate:
\[
 \Vector{ \widehat{r} }'
 =
 a \Vector{j}
 -
 b \Vector{p}
\]


For the determinant:
\[
 D(t)
 =
 \Vector{ \widehat{r} }'(t) \cdot \Vector{ r}''(t)
 =
 \left(
    a \Vector{j}
    -
    b \Vector{p}
 \right)
 \cdot
 b \Vector{p}
  =
\]
\[
 ab~
 \Vector{j} \cdot \Vector{p}
 -
 b^2
 =
 b \left(
   a \cos{t}-b
 \right)
\]

Investigating, $D(t)=0$:
\[
 a \cos{t}
 =
 b
 \quad \Leftrightarrow \quad
 \cos{t}
 =
 \frac{b}{a},
\]

Thus $D(t)$ has has two roots (counting multiplicity), if and only
if, $b \in [-a,a]$, given by:
\[
 \delta_1 = \Arccos{ \left( \frac{b}{a} \right) }
 \qquad \qquad
 \delta_2 = \pi-\delta_1
\]

This leads us to suspect to see diferent Trochoid curvature behaviour, 
in the cases: 

\begin{enumerate}
 \item $|b|>a$: $D(t)$ has fixed sign; Type 1 Trochoids.
 \item $|b|=a$: $D(t)$ has a zero, but does not change sign. 
 Cycloids or degenerated Trochoids.
 \item $|b|>a$: $D(t)$ changes sign twice; Type 2 Trochoids.
\end{enumerate}

Calculating the squared velocity:

\[
 \Vector{ r }' \cdot \Vector{ r }'
 =
 \left(
    a \Vector{i}
    +
    b \Vector{q}
  \right)
  \cdot
  \left(
    a \Vector{i}
    +
    b \Vector{q}
  \right)
  =
\]
\[
 a^2+b^2
 +2ab \Vector{i} \cdot \Vector{q}
 =
 a^2+b^2-2ab \cos{t}
\]

The Trochoid curvature is:

\[
 \kappa(t)
 =
 \frac{D(t)}{ |\Vector{r}'(t)|^3}
 =
 \frac{ b \left(
   a \cos{t}-b
 \right)
 }
 {
 (a^2+b^2-2ab \cos{t})^{3/2}
 }
\]

For the ratio:

\[
 \nu(t)
 =
 \frac{ |\Vector{r}'(t)|^2}{D(t)}
 =
 \frac{ 
    a^2+b^2-2ab \cos{t}
 }
 {
   b \left(
   a \cos{t}-b
 \right)
 }
\]

Therefore, the curvature vector, $\rho \Vector{n}$, is:

\[
 \rho \Vector{n}
 =
 \nu(t) \Vector{ \widehat{r} }'
 =
 \frac{ 
    a^2+b^2-2ab \cos{t}
 }
 {
   b \left(
   a \cos{t}-b
 \right)
 }
 \left(
    a \Vector{j}
    -
    b \Vector{p}
 \right)
\]

Thus, we obtain for the Trochoid evolute:
\[
 \Vector{c}(t)
 =
 \Vector{r}(t)+\rho \Vector{n}(t)
 =
 at\Vector{i}+a\Vector{j}
  -b \Vector{p}
  +
 \nu(t)
 \left(
    a \Vector{j}
    -
    b \Vector{p}
 \right)
 =
  at\Vector{i}
  +
  (1+\nu(t))
  \left(
    a \Vector{j}
    -
    b \Vector{p}
 \right)
\]

We observe the common factor:
\[
 1+\nu(t)
 =
 1+
 \frac{ 
    a^2+b^2-2ab \cos{t}
 }
 {
   b \left(
   a \cos{t}-b
 \right)
 }
 =
 \]
 \[
\frac{ 
    a^2+b^2-2ab \cos{t}+ba \cos{t}-b^2
 }
 {
   b \left(
   a \cos{t}-b
 \right)
 }
 =
 \frac{ 
    a^2-ab \cos{t}
 }
 {
   b \left(
   a \cos{t}-b
 \right)
 }
 =
 -\frac{a}{b}
 \cdot
 \frac{ 
    a-b \cos{t}
 }
 {
   b-a \cos{t}
 }
\]

Finally we may write the Trochoid evolute:
\[
  \Vector{c}(t)
 = 
 at\Vector{i}
 -
 \frac{a}{b}
 \cdot
 \frac{ 
    a-b \cos{t}
 }
 {
   b-a \cos{t}
 }
 \left(
    a \Vector{j}
    -
    b \Vector{p}
 \right)
\]

For this curve to be another Trochoid, the appearing factor should be constant:
\[
 \varphi(t)
 =
 \frac{ 
    a-b \cos{t}
 }
 {
   b-a \cos{t}
 }
\]

Deriving:
\[
 \varphi'(t)
 =
 \frac{ b \cos{t}- a \cos{t} }
 { (b-a \cos{t})^2}
 =
 \frac{ (b-a)\cos{t} }{ (b-a \cos{t})^2 }
 =
 0
\]

Which is satisifed, if and only if, $a=b$, being the special case of a Cycloid, $b=a$
obtaining the Cycloid evolute:
\[
  \Vector{c}(t)
  =
  at\Vector{i}
  -
   \frac{ 
    1- \cos{t}
 }
 {
   1- \cos{t}
 }
 \left(
    a \Vector{j}
    -
    a \Vector{p}
 \right)
 =
   at\Vector{i}
  -
 \left(
    a \Vector{j}
    -
    a \Vector{p}
 \right)
\]

This is a translated Cycloid, with the same half-axis
and the phase $\pi$ \emph{ahead} of the original. 
We state this in a theorem:

\begin{Theorem}
 The evolute of the Cycloid $C(a)$ is the translated Cycloid $\Vector{R}_0 + C(a,\pi)$,
 whereas the evolute of a 'true' Trochoid $T(a,b)$ is \emph{not} a Trochoid.
\end{Theorem}


