
\begin{Definition} 
   \textbf{Epicycloids and Hypocycloids.}

   Consider a fixed circle of radius $a>0$, centered in the origin,
   and a rolling circle of radius $b>0$. Rolling the latter on the
   outside of the fixed circle, results in a curve called an
   \emph{Epicycloid}, with parametrization:
\[
 \Vector{r}(t)
 =
 (a+b) \Vector{e}-b \Vector{e}(\omega t+\gamma)
\]

As for the Trochoids, we have included a phase parameter, and introduced
the angular velocity of rolling circle, $\omega=(a+b)/b$. We write $E(a,b,\gamma)$.

Changing the sign of the last term in this parametrization, 
the circle are rolling on the inside of the fixed circe
and we arrive at a socalled \emph{Hypocycloid},
$H(a,b,\gamma)$. Allowing negative $b$'s, we may write:
$H(a,b,\gamma)=E(a,-b,\gamma)$.
\end{Definition}

As to be seen in the remainder of this section, these curves presents
striking similarities with the Cycloids and Trochoids. In fact, we shall
prove that the evolute of an Epicycloid, respectively a Hypocycloid,
is again an Epicycloid, respectively a Hypocycloid. 


\begin{Definition} 
   \textbf{Epictrochoids and Hypotrochoids.}

As in the case of Cycloids and Trochoids, we extend the description to
include Trochoid like curves, with $c \in \mathbb{R}$: 
\[
 \Vector{r}(t)
 =
 (a+b) \Vector{e}-c \Vector{e}(\omega t+\gamma)
\]

For brevity, we write: $E(a,b,c,\gamma)$, respectively $H(a,b,c,\gamma)$.
\end{Definition}

We will also prove that the evolute of an Epitrochoid, respectively a Hypotrochoid,
is \emph{not} an Epitrochoid, respectively a Hypotrochoid. 


\vspace{3cm}

Dado circumferência fixo de raio $R$ e circumferência de raio $r$, rolando sem deslizar dentro deste, ou seja
com velocidadade angular:

\[
  \omega=\frac{R+r}{r}
  =
  1+\eta
\]

Epicicloid:

\[
 \Vector{r}(t)
 =
 (R+r) \Vector{e}-r \Vector{e}_\omega
\]

Derivando:

\[
 \Vector{r}'(t)
 =
 (R+r) \Vector{f}-r \omega \Vector{f}_\omega 
\]

\[
 \Vector{r}''(t)
 =
 -(R+r) \Vector{e}+r \omega^2 \Vector{e}_\omega 
\]

Transversor da primeira derivada:

\[
 \Vector{\widehat{r}}'(t)
 =
 -(R+r) \Vector{e}+r \omega \Vector{e}_\omega 
\]



Vetor de curvatura:

\[
 \rho \Vector{n}
 =
  \frac 
 {
  v(t)^2
  }
  {
    D(t)
  }
  ~
  \Vector{\widehat{r} }'(t)
  =
  \frac 
 {
  v(t)^2
  }
  {
    D(t)
  }
  ~
  \left[ -(R+r) \Vector{e}+r \omega \Vector{e}_\omega \right]
\]


Centro de Curvatura:

\[
 \Vector{c}(t)
 =
  (R+r) \Vector{e}-r \Vector{e}_\omega
  +
 \frac{ v(t)^2 }{ D(t) }
  ~
  \left[ -(R+r) \Vector{e}+r \omega \Vector{e}_\omega \right]
  =
\]
\[
 (R+r) \left[ 1- \frac{ v(t)^2 }{ D(t) } \right] \Vector{e} 
 -
 r \left[  1- \omega \frac{ v(t)^2 }{ D(t) } \right]  \Vector{e} _\omega
\]

Velocidade:

\[
 v(t)^2
 =
 (R+r)^2+r^2 \omega^2 -2(R+r) r \omega ~\Vector{f} \cdot \Vector{f}_\omega
\]

Determinante:

\[
D(t)=
 \left[ \Vector{r}'(t) \times \Vector{r}''(t)\right]
  =
 \Vector{\widehat{r} }'(t) \cdot \Vector{r}''(t)
 =
 \]
 \[ 
 \left[ -(R+r) \Vector{e}+r \omega \Vector{e}_\omega   \right]
 \cdot
 \left[   -(R+r) \Vector{e}+r \omega^2 \Vector{e}_\omega \right]
 =
 \]
 \[
 (R+r)^2+(r \omega)^2 \omega
 -
 \left\{  (R+r) r \omega^2 + r \omega (R+r) \right\} \Vector{e} \cdot \Vector{e}_\omega
\]

No Epicicloide, $r \omega=R+r$:

\[
 v(t)^2
 =
 (R+r)^2+(R+r)^2 -2(R+r) (R+r) ~\Vector{f} \cdot \Vector{f}_\omega
 =
 2(R+r)^2 \left\{ 1-\Vector{f} \cdot \Vector{f}_\omega \right\}
\]

\[
 D(t)=
 (R+r)^2+(R+r)^2 \omega
 -
 \left\{  (R+r) (R+r) \omega + (R+r) (R+r) \right\} \Vector{e} \cdot \Vector{e}_\omega
 =
\]
\[
 (R+r)^2(1+\omega) \left\{ 1-\Vector{e} \cdot \Vector{e}_\omega \right\}
\]



Juntando e usando o fato $\Vector{e} \cdot \Vector{e}_\omega=\Vector{f} \cdot \Vector{f}_\omega=\cos{\eta t}$:

\[
 \frac{v(t)^2}{D(t)}
 =
 \frac
 {
   2(R+r)^2 \left\{ 1-\Vector{f} \cdot \Vector{f}_\omega \right\}
 }
 {
   (R+r)^2(1+\omega) \left\{ 1-\Vector{e} \cdot \Vector{e}_\omega \right\}
 }
 =
\]
\[
 \frac{2}{1+\omega}
 =
 \frac{2r}{r+r\omega}
 =
 \frac{2r}{R+2r}
\]

Calculamos a evoluta do Epicicloide:

\[
 \Vector{r}_c
 =
 \Vector{r}+\frac{v(t)^2}{D(t)}\Vector{\widehat{r}}'
 =
\]
\[
 (R+r) \Vector{e}-r \Vector{e}_\omega
 +
 \frac{v(t)^2}{D(t)}
 \left[
 -(R+r) \Vector{e}+r \omega \Vector{e}_\omega 
 \right]
 =
\]

\[
 (R+r) \left\{ 1-\frac{v(t)^2}{D(t)} \right\} \Vector{e}
 -
 r \left\{ 1- \omega \frac{v(t)^2}{D(t)} \right\} \Vector{e}_\omega 
\]

Finalmente obtemos:

\[
 1-\frac{v(t)^2}{D(t)}
 =
 1-\frac{2r}{R+2r}
 =
 \frac{R+2r-2r}{R+2r}
 =
 \frac{R}{R+2r}
\]

E:

\[
  1- \omega \frac{v(t)^2}{D(t)}
  =
  1-\frac{R+r}{r} \frac{2r}{R+2r}
  =
  \frac{1}{R+2r}
  \left[ R+2r-2(R+r) \right]
  =
  -\frac{R}{R+2r}
\]

A evoluta do Epicicloide:

\[
 \Vector{r}_c(t)
 =
 (R+r) \frac{R}{R+2r} \Vector{e}
 +
 r \frac{R}{R+2r} \Vector{e}_\omega
\]

Ou seja, novamente um Epicicloide, com:

\[
 R'=\frac{R}{R+2r} R,
 \qquad
 r'=-\frac{R}{R+2r} r
\]

