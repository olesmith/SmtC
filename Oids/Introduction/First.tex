 In a point, $t=t_0$, where $\Vector{r}'(t) \neq \Vector{0}$, we have the first order aproximation:
 
 \begin{equation}
  \label{eq:taylor:1}
  \Vector{r}(t)
  \approx
  \Vector{r}(t_0)
  +
  \Vector{r}'(t_0)(t-t_0)
\end{equation}

 As the factor $t-t_0$ changes sign, this means, that the 
 curve arrives at and departures from the point $\Vector{r}(t_0)$,
 smoothly along the direction of the vector tangent, $\Vector{r}'(t_0)$. 
 Introduzing the \emph{unit tangent}, $\Vector{t}(t)$, (or simply tangent):
 
 \[
  \Vector{t}(t)=\frac{\Vector{r}'(t)}{|\Vector{r}'(t)|}
 \]
 
In coordinates:

\[
 \Vector{t}=
 \frac{\Vector{r}'}{|\Vector{r}'(t)|}
 =
 \frac{1}{\sqrt{x'^2+y'^2}}
 \begin{pmatrix}
  x'\\y'
 \end{pmatrix}
\]

By the unit normal, $\Vector{n}(t)$, we understand:

\[
 \Vector{n}=
 \frac{1}{\sqrt{x'^2+y'^2}}
 \begin{pmatrix}
  -y'\\x'
 \end{pmatrix} 
\]

We write: $\Vector{n}=\Vector{\widehat{t}}$. 
Introducing the \emph{Accompanying Coordinate System} (ACS), $(\Vector{r},\Vector{t},\Vector{n})$,
which is again positively oriented, resumed in:

\[
 \Vector{\widehat{i}}=\Vector{j},
 \qquad
 \Vector{\widehat{j}}=-\Vector{i},
 \qquad \qquad
 \Vector{\widehat{t}}=\Vector{n},
 \qquad
 \Vector{\widehat{n}}=-\Vector{t}.
 \]

 
 \Example[Ellipsis Family]{
 The family of Ellipsis with major axis $a$ and $b$ and centered in the origim, $O$, may be parametrized by:
 
 \[
 \Vector{r}(t)
 =
  \begin{pmatrix}
   x(t)\\y(t)
  \end{pmatrix}
  =
  \begin{pmatrix}
   a\cos{t}\\b \sin{t}
  \end{pmatrix},
  \qquad
  t \in [-\pi,\pi]
 \]

 Deriving:
 
 \[
  \Vector{r}'(t)
 =
  \begin{pmatrix}
   x'(t)\\y'(t)
  \end{pmatrix}
  =
  \begin{pmatrix}
   -a\sin{t}\\b \cos{t}
  \end{pmatrix} 
 \]

 We observe, that for all $t$:

 \[
  |\Vector{r}'(t)| ^2
  =
  a^2 \sin^2{t}+b^2 cos^2{t}
  \geq
  0
 \]

 That it, an ellipsis is regular in all of it's points.
 }
 
Considering a vector function, $\Vector{a}(t)$, of \emph{fixed length}: 
$\Vector{a}(t) \cdot \Vector{a}(t)=l^2$, 
deriving we obtain:

\[
 \Vector{a}(t) \cdot \Vector{a}'(t)=0,
\]

that is: the derivative vector is perpendicular to the original vector. Particularly the derivative
of unit tangent, $\Vector{t}$,
must be orthogonal to itself, that is be parallel to the (unit) normal:

\begin{equation}
  \label{eq:curvature}
 \Vector{t}'(t)
 =
 \rho \Vector{n}
\end{equation}

Taking the orthogonal complement of this equation, $\Vector{\widehat{t}'}=\widehat{\Vector{t}'}$:

\[
 \Vector{n}'(t)
 =
 \rho \Vector{t}
\]

The factor of proportionality, $\kappa=\kappa(t)$, is what we later on, shall define as the curve curvature.

In the following, the following unity vectors comes in handy:

\[
 \Vector{e}
 =
 \begin{pmatrix}
   \cos{t}\\\sin{t}
 \end{pmatrix}
 \qquad \qquad
  \Vector{f}
 =
 \begin{pmatrix}
   -\sin{t}\\\cos{t}
 \end{pmatrix}
\]


The coordinate system, $(\Vector{0},\Vector{e},\Vector{f})$ is,
as were $(\Vector{r},\Vector{e},\Vector{f})$, maintains the orientation
of $(\Vector{0},\Vector{i},\Vector{j})$. We shall need the versors, as well as
the derivatives:

\[
 \Vector{e}'
 =
  \Vector{\widehat{e}}
 =
 \Vector{f},
 \qquad\qquad
 \Vector{f}'
 =
  \Vector{\widehat{f}}
 =
 -\Vector{e}.
\]

We shall also need the orthonormal unity vectors: 
$\Vector{e}_\omega=\Vector{e}(\omega t)$ and 
$\Vector{f}_\omega=\Vector{f}(\omega t)$, the difference
from $\Vector{e}$ and $\Vector{f}$ being:

\[
  \Vector{e}_\omega'
 =
 _\omega\Vector{f}_\omega,
 \qquad \qquad
 \Vector{f}_\omega'
 =
 -_\omega\Vector{e}_\omega
\]




 \Example[Cycloids]{
 Rolling a circle of ratio $r$ on the $x$-axis, results in a curve called a Cycloid,
 parameterized by:
 
 \[
 \Vector{r}(t)
 =
  \begin{pmatrix}
   x(t)\\y(t)
  \end{pmatrix}
  =
  \begin{pmatrix}
   a(t-\cos{t})\\a(1- \sin{t})
  \end{pmatrix},
  \qquad
  t \in [-\pi,\pi]
 \]
 
 Convinientely, we write:
 
 \[
 \Vector{r}(t)
 =
   t
   \begin{pmatrix}
   a\\-1
  \end{pmatrix}
  -
  a \Vector{e}
 \]

 Deriving once:
 
 \[
   \Vector{r}'(t)
 =
   a 
   (
   \Vector{i}
    -
    \Vector{f}
   )
 \]
 We observe, that the Cycloid is regular ($\Vector{r}'(t) \neq \Vector{0}$), 
 in all points, except:
 
 \[
  t=\frac{\pi}{2} +2 p \pi, \quad p \in \mathbb{Z}
 \]
 
 For completeness:
 
 \[
  \Vector{r}''(t)
 =
 a \Vector{e}
 \]
 }
 
 
 
 